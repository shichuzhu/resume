%% start of file `template.tex'.
%% Copyright 2006-2013 Xavier Danaux (xdanaux@gmail.com).
%
% This work may be distributed and/or modified under the
% conditions of the LaTeX Project Public License version 1.3c,
% available at http://www.latex-project.org/lppl/.


\documentclass[11pt,a4paper,sans]{moderncv}        % possible options include font size ('10pt', '11pt' and '12pt'), paper size ('a4paper', 'letterpaper', 'a5paper', 'legalpaper', 'executivepaper' and 'landscape') and font family ('sans' and 'roman')

% modern themes
\moderncvstyle{banking}                            % style options are 'casual' (default), 'classic', 'oldstyle' and 'banking'
\moderncvcolor{blue}                                % color options 'blue' (default), 'orange', 'green', 'red', 'purple', 'grey' and 'black'
%\renewcommand{\familydefault}{\sfdefault}         % to set the default font; use '\sfdefault' for the default sans serif font, '\rmdefault' for the default roman one, or any tex font name
%\nopagenumbers{}                                  % uncomment to suppress automatic page numbering for CVs longer than one page

% character encoding
\usepackage[utf8]{inputenc}                       % if you are not using xelatex ou lualatex, replace by the encoding you are using
%\usepackage{CJKutf8}                              % if you need to use CJK to typeset your resume in Chinese, Japanese or Korean

% adjust the page margins
\usepackage[scale=0.91]{geometry}
%\setlength{\hintscolumnwidth}{3cm}                % if you want to change the width of the column with the dates
%\setlength{\makecvtitlenamewidth}{10cm}           % for the 'classic' style, if you want to force the width allocated to your name and avoid line breaks. be careful though, the length is normally calculated to avoid any overlap with your personal info; use this at your own typographical risks...

\usepackage{import}

\name{Shichu}{Zhu}
% \title{Curriculum Vitae}% optional, remove / comment the line if not wanted
% \address{my address, line 1, line 2, line 3, postcode}{}{}
\phone[mobile]{+1 (217) 607-6968}% optional, remove / comment the line if not wanted
\social[linkedin][www.linkedin.com/in/shichu-zhu-11aa8856]{shichu-zhu}
\social[github]{shichuzhu}
\email{shichuzhu@gmail.com}% optional, remove / comment the line if not wanted
\homepage{publish.illinois.edu/shichu-zhu/shichu-zhu}% optional, remove / comment the line if not wanted
% Photo possible

\begin{document}
\makecvtitle

% \small{Undergraduate electrical and electronic engineer completing the final year of a master's degree. Passionate about science, with strong technical, business, and interpersonal skills for working in a team and successfully completing a project.}
\vspace{-50pt}
\section{Education}
\begin{itemize}
	\item {\cventry{August 2018--Present}
	      {Computer Science}
	      {University of Illinois at Urbana-Champaign}
	      {Urbana, IL, US}
	      {Professional MS}
	      {}
	      }
	      
	\item {\cventry{August 2014--August 2018}
	      {Atmospheric Science}
	      {University of Illinois at Urbana-Champaign}
	      {Urbana, IL, US}
	      {MS}
	      {}
	      }
	      
	\item {\cventry{September 2010--July 2014}
	      {Atmospheric and Oceanic Sciences}
	      {Peking University}
	      {Beijing, China}
	      {BS}
	      {School of Physics, G.P.A. Major 3.70/4.0, Overall 3.52/4.0}
	      }
\end{itemize}

\section{Courses}
test
\section{Experience}
\vspace{6pt}
\begin{itemize}
	% \item{\cventry{July 2013--August 2013}
	%       {Position}
	%       {Employer}
	%       {Location}
	%       {position remark}
	%       {I was responsible for the administrative duties and the tidiness and general order of the site.}
	%       }
	
	\item{\cventry{August 2018--Present}
	      {Teaching Assistant}
	      {University of Illinois}
	      %   {}
	      {Urbana, IL, USA}
	      {Dept of Computer Science}
	      {\emph{CS 411 Database Systems,} Design homework questions (SQL query, ER diagrams), present tutorial lecture on web programming upon DBMS and hold office hours.}
	      }
	      
	\item{\cventry{Summer 2018}
	      {Summer Intern}
	      %   {University of Illinois}
	      {\vspace{-10pt}}
	      %   {Urbana, IL, USA}
	      {}
	      {DataSpread Group: \href{https://dataspread.github.io}{\faExternalLink dataspread.github.io}}
	      {\href{https://github.com/dataspread/dataspread-web}{\faGithub} Designed and developed the navigation browsing component, integrating front-end design and back-end database algorithm support. Achievements included augmenting ZKSpreadSheet's formula execution engine and using complex data structures such as B-Tree. Mainly developed in java/javascript with the Spring framework.}
	      }
	      
	\item{\cventry{2016--2017}
	      {Teaching Assistant}
	      %   {University of Illinois}
	      {\vspace{-10pt}}
	      %   {Urbana, IL, USA}
	      {}
	      {Dept of Atmospheric Science}
	      {\emph{ATMS 301 Thermodynamics, ATMS 201 General Physical Meteorology,} Duties included grading homework, writing quiz questions and giving mini lectures explaining homework problems.}
	      }
	      
	\item{\cventry{2014--2016}
	      {Research Assistant}
	      %   {University of Illinois}
	      {\vspace{-10pt}}
	      %   {Urbana, IL, USA}
	      {}
	      {Dept of Atmospheric Science}
	      {\href{https://github.com/joefinlon/UIOPS}{\emph{UIOPS}} Improved the theoretical mechanism of the ice clouds formation by analyzing airborne observation dataset. Technically involved developing of numerical computation and image processing in MATLAB, as well as data analysis and visualization using Python (Scipy/Pandas/Matplotlib).}
	      }
	      
	\item{\cventry{Summer 2013}
	      {Visiting Undergraduate Researcher}
	      {California Institute of Technology}
	      %   {}
	      {Pasadena, CA, USA}
	      {Dept of Planetary Science}
	      {Numerical simulation of the weather layer of Jupiter's atmosphere using GFDL's shallow water model.}
	      }
	      
	\item{\cventry{2013--2014}
	      {Undergraduate Researcher}
	      {Peking University}
	      %   {}
	      {Beijing, China}
	      {Dept of Atmospheric and Oceanic Sciences}
	      {A survey and comparison of existing numerical advection schemes in solving 1-D advection equation.}
	      }
	      
\end{itemize}

\section{Projects}
\begin{itemize}
	\item{\cventry{Fall 2017}
	      {Course project}
	      %   {University of Illinois}
	      {\vspace{-10pt}}
	      %   {Urbana, IL, USA}
	      {}
	      {CS 425 Distributed System}
	      {
		      %   \emph{Source on demand}%
		      \href{https://gitlab-beta.engr.illinois.edu/szhu28/ShichuCS425MP}{\faGitlab gitlab-beta.engr.illinois.edu/szhu28/ShichuCS425MP}
		      Implemented a gossip-style failure detector for a distributed system connected by arbitrary network topology. Programming techniques included building a utility RPC module with socket and decorator in Python.}
	      }
	      
	\item{\cventry{Fall 2017}
	      {Course project}
	      %   {University of Illinois}
	      {\vspace{-10pt}}
	      %   {Urbana, IL, USA}
	      {}
	      {CS 225 Data Structures Honor Section}
	      {
		      %   \emph{Source on demand}%
		      \href{https://github.com/shichuzhu/text_adventure_game}{\faGithub github.com/shichuzhu/text\_adventure\_game}
		      A simple terminal text adventure game built under functional programming paradigm in Clojure.}
		  }
	      
\end{itemize}

\section{Programming Languages}
\vspace{6pt}
\begin{itemize}
	\item \textbf{Proficient in:} C\texttt{++}, Python, Java.
	\item \textbf{Familiar with:} Golang, FORTRAN, MATLAB, SQL, Haskell, Clojure, javascript, \LaTeX.
\end{itemize}

% Publications from a BibTeX file without multibib
%  for numerical labels: \renewcommand{\bibliographyitemlabel}{\@biblabel{\arabic{enumiv}}}% CONSIDER MERGING WITH PREAMBLE PART
%  to redefine the heading string ("Publications"): \renewcommand{\refname}{Articles}
% \nocite{Zhu2016}
% \bibliographystyle{plain}
% \bibliography{ref}

\end{document}


%% end of file `template.tex'.
