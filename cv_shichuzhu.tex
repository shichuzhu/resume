%% start of file `template.tex'.
%% Copyright 2006-2013 Xavier Danaux (xdanaux@gmail.com).
%
% This work may be distributed and/or modified under the
% conditions of the LaTeX Project Public License version 1.3c,
% available at http://www.latex-project.org/lppl/.


\documentclass[11pt,a4paper,sans]{moderncv}        % possible options include font size ('10pt', '11pt' and '12pt'), paper size ('a4paper', 'letterpaper', 'a5paper', 'legalpaper', 'executivepaper' and 'landscape') and font family ('sans' and 'roman')

% modern themes
\moderncvstyle{banking}                            % style options are 'casual' (default), 'classic', 'oldstyle' and 'banking'
\moderncvcolor{blue}                                % color options 'blue' (default), 'orange', 'green', 'red', 'purple', 'grey' and 'black'
%\renewcommand{\familydefault}{\sfdefault}         % to set the default font; use '\sfdefault' for the default sans serif font, '\rmdefault' for the default roman one, or any tex font name
%\nopagenumbers{}                                  % uncomment to suppress automatic page numbering for CVs longer than one page

% character encoding
\usepackage[utf8]{inputenc}                       % if you are not using xelatex ou lualatex, replace by the encoding you are using
%\usepackage{CJKutf8}                              % if you need to use CJK to typeset your resume in Chinese, Japanese or Korean

% adjust the page margins
\usepackage[scale=0.91]{geometry}
%\setlength{\hintscolumnwidth}{3cm}                % if you want to change the width of the column with the dates
%\setlength{\makecvtitlenamewidth}{10cm}           % for the 'classic' style, if you want to force the width allocated to your name and avoid line breaks. be careful though, the length is normally calculated to avoid any overlap with your personal info; use this at your own typographical risks...

\usepackage{import}

\name{Shichu (Stuart)}{Zhu}
% \title{Curriculum Vitae}% optional, remove / comment the line if not wanted
% \address{my address, line 1, line 2, line 3, postcode}{}{}
\phone[mobile]{+1 (217) 607-6968}% optional, remove / comment the line if not wanted
\social[linkedin][www.linkedin.com/in/shichu-zhu-11aa8856]{shichu-zhu}
\social[github]{shichuzhu}
\email{shichuzhu@gmail.com}% optional, remove / comment the line if not wanted
\homepage{publish.illinois.edu/shichu-zhu/shichu-zhu}% optional, remove / comment the line if not wanted
% Photo possible

\begin{document}
\makecvtitle

\vspace{-40pt}
\begin{center}
	\textbf{Objective:} Software development engineer full-time position, Graduation Dec 2019.
\end{center}
\vspace{-20pt}
\section{Education}
\vspace{-5pt}
\begin{itemize}
	\item {\cventry{August 2018--Expected December 2019}
	      {Computer Science}
	      {University of Illinois at Urbana-Champaign}
	      {Urbana, IL, US}
	      {Professional MS}
	      {}
	      }
	      
	\item {\cventry{August 2014--August 2018}
	      {Atmospheric Science}
	      {University of Illinois at Urbana-Champaign}
	      {Urbana, IL, US}
	      {MS}
	      {}
	      }
	      
	\item {\cventry{September 2010--July 2014}
	      {Atmospheric and Oceanic Sciences}
	      {Peking University}
	      {Beijing, China}
	      {BS}
	      {School of Physics, G.P.A. Major 3.70/4.0, Overall 3.52/4.0}
	      }
\end{itemize}

\vspace{-5pt}
\section{Experience}
\begin{itemize}
	% \item{\cventry{July 2013--August 2013}
	%       {Position}
	%       {Employer}
	%       {Location}
	%       {position remark}
	%       {I was responsible for the administrative duties and the tidiness and general order of the site.}
	%       }
	
	\item{\cventry{August 2018--Present}
	      {Teaching Assistant}
	      {University of Illinois}
	      %   {}
	      {Urbana, IL, USA}
	      {Dept of Computer Science, CS 411 Database Systems.}
	      {Design homework questions (\underline{SQL} query, ER diagrams), present tutorial lecture on web programming with DBMS.}
	      }
	      
	\item{\cventry{Summer 2018}
	      %   {Summer Intern}
	      {Full-Stack Software Developer}
	      %   {University of Illinois}
	      {\vspace{-10pt}}
	      %   {Urbana, IL, USA}
	      {}
	      {DataSpread Group:
		      % \href{https://github.com/dataspread/dataspread-web}{\faGithub}
		      \href{https://dataspread.github.io}{\underline{dataspread.github.io}}, Prof. Aditya Parameswaran}
	      {
		      Mainly developed in \underline{java/javascript} with the Spring framework;
		      Designed and developed the navigation browsing component, integrating front-end design and back-end database algorithm support;
		      Achievements included augmenting ZK--SpreadSheet's formula execution engine and using complex data structures such as B-Tree.}
	      }
	      
	      % \item{\cventry{2016--2017}
	      %   {Teaching Assistant}
	      %   %   {University of Illinois}
	      %   {\vspace{-10pt}}
	      %   %   {Urbana, IL, USA}
	      %   {}
	      %   {Dept of Atmospheric Science}
	      %   {\emph{ATMS 301 Thermodynamics, ATMS 201 General Physical Meteorology,} Duties included grading homework, writing quiz questions and giving mini lectures explaining homework problems.}
	      %   }
	      
	\item{\cventry{2014--2017}
	      {Research Assistant}
	      %   {University of Illinois}
	      {\vspace{-10pt}}
	      %   {Urbana, IL, USA}
	      {}
	      {Dept of Atmospheric Science, Prof. Greg McFarquhar}
	      % TODO: add UIOPS repo to README of atmos-research
	      {
		      \begin{itemize}
			      \item 
			            NSF--funded research project to understand formation of ice clouds based on observed ice particle images.
			      \item Used \underline{MATLAB} programs to process the particle images, extract dimensional information and estimate their size distributions.
			      \item Statistically analyzed and visualized the distributions and their derived properties to draw scientific conclusions. The analysis was largely done in \underline{Python} (Scipy/Pandas/Matplotlib). Sample jupyter notebooks available at \href{https://github.com/shichuzhu/atmos-research}{\underline{github.com/shichuzhu/atmos-research}}.
			      \item Results presented at American Geophysical Union Fall Meeting in Dec 2016.
		      \end{itemize}
	      }
	      }
	      
	\item{\cventry{Summer 2013}
	      {Visiting Undergraduate Researcher}
	      {California Institute of Technology}
	      %   {}
	      {Pasadena, CA, USA}
	      {Dept of Planetary Science}
	      {Numerical simulation of the weather layer of Jupiter's atmosphere using GFDL's shallow water model. Original model and tuning are coded in \underline{FORTRAN}.}
	      }
	      
	\item{\cventry{2013--2014}
	      {Undergraduate Researcher}
	      {Peking University}
	      %   {}
	      {Beijing, China}
	      {Dept of Atmospheric and Oceanic Sciences}
	      {A survey and comparison of existing numerical advection schemes in solving 1-D advection equation. Implemented in \underline{FORTRAN}.}
	      }
	      
\end{itemize}

\vspace{-5pt}
\section{Projects}
\vspace{-5pt}
\begin{itemize}
	% {https://gitlab-beta.engr.illinois.edu/szhu28/ShichuCS425MP}
	\item{\cventry{Fall 2017}
	      {Course project}
	      %   {University of Illinois}
	      {\vspace{-10pt}}
	      %   {Urbana, IL, USA}
	      {}
	      {Distributed Systems,
		      \href{https://github.com/shichuzhu/sds}
		      {\underline{https://github.com/shichuzhu/sds}}}
	      {
			An simple distributed system built from scratch using \underline{Golang}.
			It includes a ping-ack SWIM failure detector, a reliable distributed file system with replica control, and a stream-processing engine with a naive scheduler.
	      }}
	      
	\item{\cventry{Fall 2017}
	      {Course project}
	      %   {University of Illinois}
	      {\vspace{-10pt}}
	      %   {Urbana, IL, USA}
	      {}
	      {Data Structures Honor Section,
		      \href{https://github.com/shichuzhu/text_adventure_game}{\underline{github.com/shichuzhu/text\_adventure\_game}}
	      }
	      {
		      %   \emph{Source on demand}%
		      A simple terminal text adventure game built under functional programming paradigm in \underline{Clojure}.}
	      }
\end{itemize}

\vspace{-5pt}
\section{Courses}
\vspace{-5pt}
\begin{itemize}\small
	\item \textbf{Theory:}
	      Algorithms and Data Structures,
	      Applied Numerical Methods [\emph{A\texttt{+}}],
	      Introduction to Computation [\emph{A\texttt{+}}],
		Introduction to Data Mining,
		Artificial Intelligence,
	      Linear Algebra,
	      Probability and Statistics.
	\item \textbf{System:}
	      Applied Cloud Computing (Python),
	      Communication Networks (C\texttt{++}),
	      Database Systems,
	      Distributed Systems (Python, Go)[\emph{A\texttt{+}}],
	      Programming Languages (Haskell),
	      System Programming (C).
\end{itemize}

\vspace{-5pt}
\section{Programming Skills}
\vspace{-5pt}
\begin{itemize}
	\item \textbf{Proficient in:}
	      C\texttt{++}, 
	      Java,
	      Python.
	\item \textbf{Familiar with:}
	      C,
	      Clojure,
	      FORTRAN,
	      Go,
	      Haskell,
	      MATLAB,
	      SQL,
	      \LaTeX,
	      javascript.
	\item \textbf{Frameworks \& Tools:}
	      React,
	      gRPC,
	      Django,
	      Jupyter notebook,
	      Node.js, 
	      Spring.
	\item \textbf{Programming Contests:}
	      \href{https://icpc.cs.illinois.edu/halloffame.html}{Illinois Programming League} (IPL) Rank 5 (Season 3), Rank 9 (Season 2).
\end{itemize}

% Publications from a BibTeX file without multibib
%  for numerical labels: \renewcommand{\bibliographyitemlabel}{\@biblabel{\arabic{enumiv}}}% CONSIDER MERGING WITH PREAMBLE PART
%  to redefine the heading string ("Publications"): \renewcommand{\refname}{Articles}
% \nocite{Zhu2016}
% \bibliographystyle{plain}
% \bibliography{ref}

\end{document}


%% end of file `template.tex'.
